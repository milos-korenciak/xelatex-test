\documentclass[11pt,a4paper]{article}
\usepackage[a4paper]{geometry}
\usepackage{graphicx}
\usepackage{booktabs}
\usepackage{fontspec} % to use a custom font (e.g., Roboto). To use with lualatex instead of pdflatex
\setmainfont{Roboto}
\usepackage{lastpage} % to use the last page number in footer
\usepackage{datetime} % to get months as a words, not numbers
%\usepackage[italian,english]{babel} % to get some another languages later on (using \selectlanguage{italian})
\usepackage{fancyhdr}
\pagestyle{fancy}  % to have header and footer
\usepackage{grffile} % to manage fancy characters in figure file names (i.e., underscores)

% the next 5 lines are to set the figure positioning. Figures in spare pages are centered to
% cover all the blank space. With these settings they are forced to align to the top of the page
\makeatletter
\setlength\@fptop{0pt} 
\setlength\@fpsep{30pt plus 0fil} 
\setlength\@fpbot{0pt}
\makeatother

\setlength{\abovetopsep}{1ex} % to set up the separation distance between the caption and the table

\setlength\topmargin{-40pt} % Top margin
\setlength\headheight{20pt} % Header height
%\setlength\textwidth{7.0in} % Text width
\setlength\textheight{9.5in} % Text height
%\setlength\oddsidemargin{-30pt} % Left margin
%\setlength\evensidemargin{-30pt} % Left margin (even pages) - only relevant with 'twoside' article option

\fancyhf{}
\fancyhead[R]{climData}
\fancyhead[L]{\includegraphics[width=2cm]{${logo_filepath}}}
\fancyfoot[R]{\thepage / \pageref{LastPage}}
\fancyfoot[L]{Chapter \nouppercase{\leftmark} \\ {\textcopyright} Solargis, {\monthname} \the\year}

%opening
\title{Solargis$$^{\textcopyright}$$ Report}
\author{Solargis\texttrademark  s.r.o}

\begin{document}

\maketitle
\textsc{\LARGE University Name}\\[1.5cm]
\thispagestyle{empty}

\newpage
\setcounter{page}{1}

\section{Forecasting methodology and assessment details}
The Solargis\texttrademark forecasting technology comprises various cutting-edge methodologies to predict solar radiation and other meteorological variables. Our technology provides worldwide forecasting solutions for time horizons ranging from few minutes up to 10 days ahead with sub-hourly resolution and several updates a day. From 10 minutes to 6 hours ahead, the solar radiation prediction is performed based on both time casting of near-real-time satellite-derived images and numerical weather prediction (NWP) models. For time horizons beyond 6 hours, NWP models are the only current method providing reliable estimates.

\end{document}